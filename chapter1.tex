\chapter{INTRODUCTION}



\begin{singlespace}
\epigraph{One can even shout out through refuse\ldots}{\hfill---Kurt Schwitters, \textit{Kurt Schwitters}, 1985}
\end{singlespace}



%%% Quotation examples:
Stav says that \quotes{[t]hey forget about it and don’t think about all the time and energy and money put into disposing of it} \citep[as cited in][]{navarro2015followingtrash}. 

\citet[11]{banash2013collage} draws attentions to the advancements in the recent history of humankind by stating that \quotes{over the course of the twentieth century, the twin developments of mass production and mass media in the capitalist economies of the [western countries] completed a total transformation of everyday life, reorienting almost every activity toward consumption.} 

\quotes{The phenomenon of waste comes clearly into focus not merely as a by-product of manufacturing processes, but rather as an integral element in cycles of production and consumption} \citep[ix]{pye2010trashculture}.

Anyone can encounter with trash in the crowded urban areas \quotes{as well as the remotest corners of the world} \citep[16]{cerny1996recycled}.

%%% Footnote example:
Trash is in the streets, in people’s home, in the sea that people swim, rotating around the globe\footnote{Satellite discards are disposed to the atmosphere and they rotate around the globe like satellites.}. Even if trash is tried to move away from people’s habitat, it is as close as the nearest waste bin.

%%% Block quotation example:
The authors of the book \textit{Rubbish: the Archeology of Garbage} give clearer definition of some these words:

\begin{quote}
\textit{Trash} refers specifically to discards that are at least theoretically dry ---newspapers, boxes, cans, and so on. \textit{Garbage} refers technically to wet discards ---food remains, yard waste, and offal. \textit{Refuse} is an inclusive term for both the wet discards and the dry. \textit{Rubbish} is even more inclusive: It refers to all refuse plus construction and demolition debris. The distinction between wet and dry garbage was important in the days when cities slopped garbage to pigs, and needed to have the wet material separated from the dry; it eventually became irrelevant, but may see a revival if the idea of composting food and yard waste catches on. \citep[9]{rathje1992rubbish}
\end{quote}
